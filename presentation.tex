\documentclass{beamer}

% ========   PACKAGES    ========

\usepackage{xcolor}
\definecolor{unipdred}{RGB}{155,0,20}
\usepackage{commands}
\usepackage{intervalc}

% ======== CONFIGURATION ========

\usetheme{CambridgeUS}
\usecolortheme{beaver}

% ========    CONTENT    ========

\begin{document}

\section{Introduction}
\subsection{Motives}
\begin{frame}[Static analysis in modern software]
  Lorem ipsum.
\end{frame}
\subsection{Abstract interpretation}
\begin{frame}[History]
  Lorem ipsum.
\end{frame}
\subsection{Previous work}
\begin{frame}[Precise interval analysis revised]
  Lorem ipsum.
\end{frame}
\subsection{Our approach}
\begin{frame}[semantics and domain analysis]
  Lorem ipsum.
\end{frame}

\section{Framework}
\subsection{The \(\imp\) language}
\begin{frame}[\(\imp\) grammar]
\begin{align*}
  \expr \ni \com[e] ::= & \; \var \in I % \mid \tru \mid \ff 
                          \mid \var := k \mid \var := \var[y] + k % \mid \var := \var[y] - k 
  \\
  % \imp\starless \ni \com[D] ::= & \; \com[e] \mid \com[D + D] \mid \com[D ; D] \\
  \imp \ni \com[C] ::= & \; \com[D] \mid \com + \com \mid \com ; \com \mid \com^* \mid \fix{\com}
\end{align*}
\end{frame}
\begin{frame}[\(\imp\) semantics]

\end{frame}
\subsection{Concrete domain}
\subsection{Non-relational collecting analysis}
\subsection{Interval analysis}

\section{Bounded analysis}
\subsection{Bounded interval analysis}
\subsection{Bounded Non-relational collecting}

\section{Conclusions}
\subsection{Our findings}
\begin{frame}[Abstraction chain]
  \begin{figure}
    \centering
    \usetikzlibrary{arrows.meta}
    \begin{tikzpicture}[, >=stealth]
      % Nodes
      \node (dom) {\(\dom\)};
      \node (bcnr) [below=of dom] {\(\bCnr\)};
      \node (ph1) [below=of bcnr] {};
      \node (inte) [left=of ph1] {\(\inte\)};
      \node (bbcnr) [right=of ph1] {\(\bbCnr{\low}{\upp}\)};
      \node (binte) [below=of inte] {\(\binte{\low}{\upp}\)};
      \node (btbcnr) [below=of bbcnr] {\(\btbCnr{\low}{\upp}\)};

      \node at (-4,-3) (decidable)   {{\color{codegreen}\(\mathsf{decidable}\)}};
      \node at (3, -2) (unknown)     {{\color{airforceblue}\(\mathsf{unknown}\)}};
      \node at (2, .5) (undecidable) {{\color{red}\(\mathsf{undecidable}\)}};

      % divisor lines
      \draw[red, opacity=.4] (-3,1) edge[out=-40,in=180] (3,-1);
      % \draw (description) edge[out=180,in=0,->] (text);
      \draw[red, opacity=.4] (-3,-1) edge[out=-40,in=180] (3,-3.5);
      
      % Arrows
      \path
      (dom) edge[->, bend right=10] node[left]{$\abstr$} (bcnr)
      (bcnr) edge[->, bend right=10] node[right]{$\concr$} (dom)
      (bcnr) edge[->, bend right=10] node[left]{$\dabstr$} (inte)
      (inte) edge[->, bend right=10] node[right]{$\dconcr$} (bcnr)
      (bcnr) edge[->, bend right=10] node[left]{$\sabstr[\low,\upp]$} (bbcnr)
      (bbcnr) edge[->, bend right=10] node[right]{$\sconcr[\low,\upp]$} (bcnr)
      (inte) edge[->, bend right=10] node[left]{$\dabstr[\low,\upp]$} (binte)
      (binte) edge[->, bend right=10] node[right]{$\dconcr[\low,\upp]$} (inte)
      (bbcnr) edge[->, bend right=10] node[left]{$\tabstr[\low,\upp]$} (btbcnr)
      (btbcnr) edge[->, bend right=10] node[right]{$\tconcr[\low,\upp]$} (bbcnr);

    \end{tikzpicture}
    \caption{Abstractions chain we build throughout
      Chapter~\ref{ch:abstractdomains}}\label{fig:abstrchain}
  \end{figure}
\end{frame}
\subsection{Future work}

\end{document}
